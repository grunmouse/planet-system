\section{Звезда}
\subsection{Свойства звезды}
\paragraph{Масса} - нужна для расчёта орбит.
\paragraph{Светимость} - нужна для расчёта зоны обитаемости.
\paragraph{Радиус} - нужен для промежуточных расчётов и классификации.
\paragraph{Эффективная тепература} - нужна для промежуточных расчётов и классификации.
\paragraph{Спектр и классификация} - нужны для учёта других параметров и для правильного описания.

\subsection{Rонстанты и вводимые обозначения}
$$L_\odot = 3.827\cdot 10^{26} \mbox{Вт}$$
$$l=\frac{L}{L_\odot};\; m=\frac{M}{M_\odot}.$$
\subsection{Взаимосвязь параметров}


\subsubsection{Температура-светимость}
\begin{equation}
L=4\pi R^2\sigma T_E^4.
\end{equation}
$\sigma$ - постоянная Стефана-Больцмана,
$T_E$ - эффективная температура.

\subsubsection{Масса-светимость}

\begin{equation}
	l = m^a.
\end{equation}

\begin{equation}
\begin{array}{ll}
	l \approx 0.32m^{2.3}, & (m<0.43), \\
	l \approx m^{4}, & (0.43<m<2), \\
	l \approx 1.5m^{3.5}, & (2<m<20), \\
	l \approx 3200m, & (m>20).
\end{array}
\end{equation}

Для звёзд с $m<0.43$ основным механизмом переноса является конвекция. 
Вариант $2<m<20$ - относится к звёздам главной последовательности и не применим к красным гигантам и белым карликам.
Если достигнут предел Эддингтона - $a=1$.

Для белых карликов - не существует зависимости масса-светимость.


\subsubsection{Предел Эддингтона}

\begin{equation}L_{edd} = \frac{4\pi GM m_p c}{\sigma_T}\end{equation}
\begin{equation}L_{edd} = 10^{38}\frac{M}{M_\odot}\mbox{эрг/с}\end{equation}
\begin{equation*}L_{edd} = 10^{31}\frac{M}{M_\odot}\mbox{Вт}\end{equation*}
\begin{equation}\frac{L_{edd}}{L_\odot} = 2,613\cdot 10^{4}\frac{M}{M_\odot}\end{equation}

\subsubsection{Масса-радиус}

Для белых карликов



\subsubsection{Главная последовательность}
Массы $0.08<m<120-200$.

