\section{Планета}

\subsection{Параметры планеты}

\subsubsection{Эффективная температура}
\begin{equation}
	\varepsilon = \frac{L}{4\pi r^2}.
\end{equation}
\begin{align*}
	\text{где}\;  
	\varepsilon & - \text{солнечная постоянная,}\\
	r & - \text{расстояние до солнца.}
\end{align*}

\begin{equation}
	T_E = \sqrt[4]{\frac{\varepsilon \left( 1- a \right) }{4\sigma}},
\end{equation}
\begin{align*}
	\text{где}\;  
	T_E & - \text{эффективная температура,}\\
	\sigma & - \text{константа Стефана-Больцмана,}\\
	a & - \text{геометрическое альбедо.}
\end{align*}

\subsubsection{Идеализированный парниковый эффект}
https://ru.qwe.wiki/wiki/Idealized\_greenhouse\_model

\begin{equation}
	\Phi_\downarrow = \frac{\varepsilon}{4}
\end{equation}
\begin{align*}
	\text{где}\;  
	\Phi_\downarrow & - \text{плотность потока входящего излучения,}\\
	\varepsilon & - \text{солнечная постоянная.}
\end{align*}

\begin{equation}
	\Phi_\uparrow = \alpha \sigma T_a^4 + (1-\alpha)\sigma T_s^4;
\end{equation}
\begin{align*}
	\text{где}\;  
	\Phi_\uparrow & - \text{плотность потока исходящего излучения,}\\
	\varepsilon & - \text{солнечная постоянная.}
\end{align*}

Радиационный баланс верхней атмосферы
\begin{equation}
	-\frac{1}{4} \varepsilon (1 - a) + \alpha \sigma T_a^4 + (1-\alpha)\sigma T_s^4 = 0.
\end{equation}
\begin{align*}
	\text{где}\;  
	\alpha & - \text{излучательная и поглощательная способность атмосферы,}\\
	T_a & - \text{температура верхних слоёв атмосферы,}\\
	T_s & - \text{температура поверхности.}
\end{align*}


Радиационный баланс поверхности
\begin{equation}
	\frac{1}{4} \varepsilon (1 - a) + \alpha \sigma T_a^4 - \sigma T_s^4 = 0.
\end{equation}

$$2\alpha \sigma T_a^4 - \alpha \sigma T_s^4 = 0;$$
$$ T_a^4 =  \frac{1}{2} T_s^4;$$

$$\Phi_\uparrow = \alpha \sigma \frac{1}{2} T_s^4 + (1-\alpha)\sigma T_s^4;$$
$$\Phi_\uparrow =  ( 1-\frac{\alpha}{2})\sigma T_s^4;$$

\begin{equation}
	T_E^4 =  ( 1-\frac{\alpha}{2}) T_s^4.
\end{equation}
